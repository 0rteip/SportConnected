\documentclass[12pt]{report}

\usepackage{alltt, fancyvrb, url}
\usepackage{titlesec}	% Remove Chapther N*
\titleformat{\chapter}[display]
  {\normalfont\bfseries}{}{0pt}{\Huge}

\usepackage{graphicx}
\usepackage[utf8]{inputenc}
\usepackage{float}
\usepackage{hyperref}
\usepackage[table,xcdraw]{xcolor}

% Questo commentalo se vuoi scrivere in inglese.
\usepackage[italian]{babel}

\usepackage[italian]{cleveref}

% \usepackage[none]{hyphenat} % Disabilita la divisione delle parole

\usepackage{tabularx}

\usepackage[left=2.5cm, right=2.5cm, top=2.5cm, bottom=3cm]{geometry}

\title{Elaborato per il corso di\\``Basi di Dati''\\[0.3in]
	\large \it Progetto di una base di dati per la gestione di dati di un'applicazione
}

\author{Pietro Ventrucci\\\href{mailto:pietro.ventrucci@studio.unibo.it}{pietro.ventrucci@studio.unibo.it}\\00001031205}
\date{\today}

\begin{document}

\maketitle

\tableofcontents

\chapter{Analisi dei requisiti}

Si vuole creare una basi di dati a supporto dell'applicazione \emph{SportConnected} per gestire le attività
svolte dagli utenti. La base di dati immagazzina quindi informazioni su utenti e relativa attività al fine
di poter fornire una consultazione comoda per visualizzare i progressi del proprio allenamento o degli altri.

\section{Intervista}
Una persona si registra e viene identificata univocamente nel sistema da un codice, anche un nome, cognome, 
data di nascita; opzionalmente una foto profilo, una biografia, città e provincia, peso, altezza. 
Si può anche indicare il tipo di attrezzatura utilizzata (tipologia di bici o marca di scarpe).
Il tipo di esercizio, la durata dell'attività, il genere, gli anni, il peso e l'altezza possono servire per
calcolare le calorie spese, la potenza o altri dati utili. Un utente può seguire o essere seguito da altri 
utenti (potrebbero essere suggeriti in base ad amicizie comuni). Ogni utente ha un registro delle proprie 
attività. Possono anche essere salvati dei percorsi o dei segmenti, relativi ad attività di altri utenti, 
o proprie attività i quali vengono salvati sotto forma di file .gpx.

Un percorso rappresenta l'intero itinerario dell'attività e può contenere o meno dei segmenti, mentre un 
segmento è una sezione con determinate caratteristiche, quali salita, discesa, tratto difficile, 
acque libere, ... Ogni tratto completato ha un tempo di percorrenza associato. Le attività hanno i 
propri segmenti e le proprie classifiche.

Possono inoltre essere creati segmenti dagli utenti, a condizione che 
il segmento sia presente nel percorso svolto e che non ne siano presenti altri simili, potrà essere 
impostato come pubblico o privato. Possono essere assegnati dei riconoscimenti all' utente in base ai 
tempi raggiunti sui segmenti, titoli come: più veloce di tutti, top 10, record personale...

Le attività possono essere caricate, decidendo la visibilità, il titolo, una descrizine, il tipo si sport, 
un percorso, un giorno, distanza totale, qualche foto, dislivello, velocità media, tempo, calorie bruciate, altitudine... 
Vengono visualizzati anche i relativi segmenti del percorso. Possono poi essere commentate da altri 
utenti in grado di assegare anche un apprezzamento.

\section{Estrazione dei concetti principali}

\subsection*{Glossario dei termini}
Si estraggono dall'intervista i concetti principali, fornendone una breve descrizione, eventuali sinonimi e
relazioni con altri concetti.

\begin{table}[h!]
    \centering
    \renewcommand{\arraystretch}{1.5} % Altezza delle righe
    \begin{tabularx}{\textwidth}{
        >{\raggedright\arraybackslash}p{\dimexpr.2\linewidth-2\tabcolsep-1.3333\arrayrulewidth}% column 1
        >{\raggedright\arraybackslash}p{\dimexpr.45\linewidth-2\tabcolsep-1.3333\arrayrulewidth}% column 2
        >{\raggedright\arraybackslash}p{\dimexpr.15\linewidth-2\tabcolsep-1.3333\arrayrulewidth}% column 3
        >{\raggedright\arraybackslash}p{\dimexpr.2\linewidth-2\tabcolsep-1.3333\arrayrulewidth}% column 4
        }
    \arrayrulecolor[HTML]{BDBFC3}
    \rowcolor[HTML]{cef3fe} 
    \textbf{Termine} & \textbf{Descrizione} & \textbf{Sinonimi} & \textbf{Collegamenti} \\
    Utente & Persona che si registra all'applicazione. & Persona & Attività\\ \hline
    Attrezzatura & Materiale utilizzato da un utente per svolgere le proprie attività. & Bici, Scarpe & Utente, Attività \\ \hline 
	Attività & L'insieme di tutte le informazioni dell'attività sportiva svolta da un utente. & & Utente, Percorso\\ \hline
    Percorso & Intera tratta percorsa in un'attività. & Itinerario & Attività, Segmento\\ \hline
    Segmento & Sezione del percorso con determinate caratteristiche. Possono essere creati dagli utenti. & Tratto & Percorso, Utente \\ \hline
    Riconoscimento & Titolo assegnato ad un utente sul segmento in base al tempo di impiegato. & Titolo & Utente, Segmento\\ \hline
	Commento & Messaggio lasciato da utenti inerente all'attività svolta da altri utenti. & & Utente, Attività \\
    \end{tabularx}
\end{table}

\subsection*{Ristrutturazione dei requisiti}
A seguito dell'analisi dei requisiti e all'estrazione dei concetti principali si procede eliminando omonimie e
riformulando la richiesta in maniera che risulti più chiara e semplice da interpretare.


\end{document}