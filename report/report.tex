\documentclass[12pt]{report}

\usepackage{alltt, fancyvrb, url}
\usepackage{titlesec}	% Remove Chapther N*
\titleformat{\chapter}[display]
  {\normalfont\bfseries}{}{0pt}{\Huge}

\usepackage{graphicx}
\usepackage[utf8]{inputenc}
\usepackage{float}
\usepackage{hyperref}
\usepackage[table,xcdraw]{xcolor}

% Questo commentalo se vuoi scrivere in inglese.
\usepackage[italian]{babel}

\usepackage[italian]{cleveref}

\usepackage{changepage}
% \usepackage[none]{hyphenat} % Disabilita la divisione delle parole

\usepackage{tabularx}

\usepackage[left=2.5cm, right=2.5cm, top=2.5cm, bottom=3cm]{geometry}

\title{Elaborato per il corso di\\``Basi di Dati''\\[0.3in]
	\large \it Progetto di una base di dati per la gestione di dati di un'applicazione
}

\author{Pietro Ventrucci\\\href{mailto:pietro.ventrucci@studio.unibo.it}{pietro.ventrucci@studio.unibo.it}\\00001031205}
\date{\today}

\begin{document}

\maketitle

\tableofcontents

\chapter{Analisi dei requisiti}

Si vuole creare una basi di dati a supporto dell'applicazione \emph{SportConnected} per gestire le attività
svolte dagli utenti. La base di dati immagazzina quindi informazioni su utenti e relativa attività al fine
di poter fornire una consultazione comoda per visualizzare i progressi del proprio allenamento o degli altri.

\section{Intervista}
Una persona si registra e viene identificata univocamente nel sistema da un codice, anche un nome, cognome, 
data di nascita; opzionalmente una foto profilo, una biografia, città e provincia, peso, altezza. 
Si può anche indicare il tipo di attrezzatura utilizzata (tipologia di bici o marca di scarpe).
Il tipo di sport, la durata dell'attività, il genere, gli anni, il peso e l'altezza possono servire per
calcolare le calorie spese, la potenza o altri dati utili. Un utente può seguire o essere seguito da altri 
utenti (potrebbero essere suggeriti in base ad amicizie comuni). Ogni utente ha un registro delle proprie 
attività. Possono anche essere salvati dei percorsi o dei segmenti, relativi ad attività di altri utenti, 
o proprie attività i quali vengono salvati sotto forma di file .gpx.

Un percorso rappresenta l'intero itinerario dell'attività e può contenere o meno dei segmenti, mentre un 
segmento è una sezione con determinate caratteristiche, quali salita, discesa, tratto difficile, 
acque libere\dots Ogni tratto completato ha un tempo di percorrenza associato. Le attività hanno i 
propri segmenti e le proprie classifiche.

Possono inoltre essere creati segmenti dagli utenti, a condizione che 
il segmento sia presente nel percorso svolto e che non ne siano presenti altri simili, potrà essere 
impostato come pubblico o privato. Possono essere assegnati dei riconoscimenti all' utente in base ai 
tempi raggiunti sui segmenti, titoli come: più veloce di tutti, top 10, record personale\dots

Le attività possono essere caricate, decidendo la visibilità, il titolo, una descrizine, il tipo si sport, 
un percorso, un giorno, distanza totale, dislivello, velocità media, tempo, calorie bruciate, altitudine\dots 
Vengono visualizzati anche i relativi segmenti del percorso. Possono poi essere commentate da altri 
utenti in grado di assegare anche un apprezzamento.

\section{Estrazione dei concetti principali}

\subsection*{Glossario dei termini}
Si estraggono dall'intervista i concetti principali, fornendone una breve descrizione, eventuali sinonimi e
relazioni con altri concetti.

\begin{table}[h!]
    \centering
    \renewcommand{\arraystretch}{1.5} % Altezza delle righe
    \begin{tabularx}{\textwidth}{
        >{\raggedright\arraybackslash}p{\dimexpr.2\linewidth-2\tabcolsep-1.3333\arrayrulewidth}% column 1
        >{\raggedright\arraybackslash}p{\dimexpr.47\linewidth-2\tabcolsep-1.3333\arrayrulewidth}% column 2
        >{\raggedright\arraybackslash}p{\dimexpr.15\linewidth-2\tabcolsep-1.3333\arrayrulewidth}% column 3
        >{\raggedright\arraybackslash}p{\dimexpr.18\linewidth-2\tabcolsep-1.3333\arrayrulewidth}% column 4
        }
    \arrayrulecolor[HTML]{BDBFC3}
    \rowcolor[HTML]{cef3fe} 
    \textbf{Termine} & \textbf{Descrizione} & \textbf{Sinonimi} & \textbf{Relazioni} \\
    Utente & Persona che si registra all'applicazione. Può caricare le sue attività o salvare percorsi e segmenti fatti da altri utenti. Può interagire con altri utenti. & Persona & Attrezzatura, Attività, Percorso, Segmento\\ \hline
    Attrezzatura & Materiale utilizzato da un utente per svolgere le proprie attività. & Bici, Scarpe & Utente, Attività \\ \hline 
	Attività & L'insieme di tutte le informazioni dell'attività sportiva svolta da un utente. & Sport & Utente, Percorso\\ \hline
    Sport & Tipologia di attività effettuata. & Attività & Attività, Segmento\\ \hline
    Percorso & Intera tratta percorsa in un'attività. & Itinerario & Attività, Segmento\\ \hline
    Segmento & Sezione del percorso con determinate caratteristiche. Possono essere creati dagli utenti. & Tratto & Percorso, Utente \\ \hline
    Riconoscimento & Titolo assegnato ad un utente sul segmento in base al tempo di impiegato. & Titolo & Utente, Segmento\\ \hline
	Commento & Messaggio lasciato da utenti inerente all'attività svolta da altri utenti. & & Utente, Attività \\
    \end{tabularx}
\end{table}

\subsection*{Ristrutturazione dei requisiti}
A seguito dell'analisi dei requisiti e all'estrazione dei concetti principali si procede eliminando omonimie e
riformulando la richiesta in maniera che risulti più chiara e fruibile per la realizzazione della base di dati.

\vspace{12pt}

Per ogni \textbf{utente} vengono salvati un codice, nome, cognome, data di nascita e opzionalmente una foto profilo,
biografia, città e provincia, peso, altezza, \textbf{attrezzatura} utilizzata. Un utente può seguire o essere seguito 
da altri utenti. Ad ogni utente vengono associate le relative \textbf{attività} e può salvare \textbf{percorsi} o 
\textbf{segmenti} di altri utenti. Le informazioni sull'utente unite a quelle dell'\textbf{attività} possono essere
utilizzare per calcolare calorie bruciate, potenza\dots

L'\textbf{attrezzatura} è indicabile scegliendo tipologia di bici o marca di scarpe.

Le \textbf{attività} devono specificare titolo, visibilità, \textbf{percorso}, giorno, distanza totale,
tempo, velocità media\dots In base allo \textbf{sport} effettuato hanno più o meno dati rilevanti. Le attività rese pubbliche 
possono ricevere \textbf{commenti} e like da altri \textbf{utenti}.

Lo \textbf{sport} relativo ad un'attività può essere caratterizzato o meno da dati rilevanti salvati, come dislivello o altitudine.

Il \textbf{percorso} è rappresentato da una traccia .gpx e da possibili \textbf{segmenti}.

Un \textbf{segmento} è rappresentato da una traccia .gpx, una determinata caratteristica, un tempo di
percorrenza associato in attività, visibilità pubblica o privata. Possono essere percorsi più volte durante un singolo
percorso. I segmenti sono relativi al tipo di \textbf{sport} e possono 
essere creati dagli \textbf{utenti} se presenti nel \textbf{percorso} caricato in un attività e non ne sono presenti 
di simili.

Un \textbf{riconoscimento} può essere assegnato ad un \textbf{utente} in base al tempo impiegato su un determinato 
\textbf{segmento}.

I \textbf{commenti} sono riferiti ad un'\textbf{attività} e scritti da un \textbf{utente}, possono ricevere like.

\chapter{Progettazione concettuale}
\section{Schema scheletro}

\begin{figure}[H]
    \includegraphics[width=\textwidth]{scheletro.png}
    \centering
    \caption{\emph{Schema scheletro con le principali entità.}}
    \label{img:schema_scheletro}
\end{figure}

\section{Raffinamenti proposti}

\begin{figure}[H]
    \includegraphics[width=\textwidth]{utente.png}
    \centering
    \caption{\emph{Schema raffinamento utente.}}
    \label{img:schema_utente}
\end{figure}

Per un utente, identificato da un codice univoco, è rappresentato il nome, il cognome,
la data di nascita e opzionalmente il percorso ad una foto profilo caricata, una biografia,
la residenza composta da città e provincia, il peso e l'altezza.
La auto-riferenziazione in utente serve per esprimere il concetto di utenti seguiti e che seguono.

\begin{figure}[H]
    \includegraphics[width=\textwidth]{attrezzatura.png}
    \centering
    \caption{\emph{Schema raffinamento attrezzatura.}}
    \label{img:schema_attrezzatura}
\end{figure}

L'entità attrezzatura identificata da un nickname è univoca e rappresenta una generalizzazione delle due tipologie 
di attrezzatura disponibili: bicicletta e scarpe, viene quindi utilizzata la specializzazione in queste due sottoclassi. 
Devono esistere nel DB delle entità che contengano tutte le tipologie di bici o le marche di scarpe associabili a bici 
e scarpe. L'attributo modello viene lasciato ad attrezzatura in quanto sia che si tratti di scarpe che di bici andrà specificato.

\begin{figure}[H]
    \includegraphics[width=\textwidth]{attivita2.png}
    \centering
    \caption{\emph{Schema raffinamento attività.}}
    \label{img:schema_attivita}
\end{figure}

Ogni attività è associata ad uno sport che può essere di varie tipologie, per questo motivo la 
generalizzazione è parziale ed esclusiva (esistono altri sport) e si specializza in tre 
entità con attributi specifici ad esse, un percorso, il quale è univoco, e contiene
o meno dei segmenti. Possono ricevere commenti scritti da utenti ed ogni commento ha un codice
univoco riferito all'attività dato il grnade numero di commenti che potrebbero essere scritti
il numero potrebbe crescere troppo ed è preferibile tenerlo basso.

\begin{figure}[H]
    \includegraphics[width=\textwidth]{segmento.png}
    \centering
    \caption{\emph{Schema raffinamento segmento.}}
    \label{img:schema_segmento}
\end{figure}

Il segmento è un'entità che contiene tutti i segmenti esistenti, mentre il tempo di percorrenza
del segmento durante l'attività viene registro tramite segmento percorso. 

In questa maniera non riesco ad esprimere il concetto di segmento appartenente ad uno sport.

\begin{figure}[H]
    \includegraphics[width=\textwidth]{sport_segmento.png}
    \centering
    \caption{\emph{Schema raffinamento caratteristica segmento.}}
    \label{img:schema_sport_segmento}
\end{figure}

Per esprimere il vincolo precedente collego tramite associazione sport e segmento.

\begin{figure}[H]
    \includegraphics[width=\textwidth]{segmenti_attivita.png}
    \centering
    \caption{\emph{Schema raffinamento segmenti e attività.}}
    \label{img:schema_seg_att}
\end{figure}

Un percorso è associato ad un'attività e può contenere dei segmenti, mentre il tempo effettivo
di percorrenza viene associato all'attività, identificato da un id di percorrenza
e dal segmento stesso percorso, così è possibile percorrere un segmento più volte durante una'attività,
mentre il percorso contine i segmenti una sola volta. 

Rimane inespresso il vincolo per cui
i segmenti contenuti nel percorso e quelli effettivamente eseguiti debbano essere gli stessi.

\begin{figure}[H]
    \includegraphics[width=\textwidth]{perc_seg.png}
    \centering
    \caption{\emph{Schema raffinamento percorso e segmento}}
    \label{img:schema_perc_seg}
\end{figure}

I percorsi e i segmenti hanno entrambi una traccia e si riferiscono a tratti di strada od altro percorsi
durante un'attività, per questo motivo vengono generalizzati dall'entità traccia, totale ed esclusiva,
dove l'attributo traccia rappresenta il percorso in memoria nel quale viene salvata la traccia .gpx.


\section{Schema concettuale finale}

\newgeometry{margin=0cm}
    \begin{figure}[p]
        \centering
        \includegraphics[width=\paperwidth]{definitivo2-1.png}
        
    \end{figure}
    
    \begin{figure}[p]
        \centering
        \includegraphics[width=\paperwidth]{definitivo2-2.png}
    \end{figure}
    
\restoregeometry

\end{document}